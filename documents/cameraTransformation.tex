\documentclass{article}
\usepackage{amsmath}
\usepackage{amssymb}

\begin{document}

\title{Space Visualizations}
\author{Jonah Schwartzman}
\date{\today}

\maketitle

\section{Introduction}
This document is not real.

\section{Camera3D}
A Camera3D represents a viewer in 3D space, with a focal length to the view plane. The main function of a Camera3D is to convert from a point in world view to a 2D point on the view plane.
\begin{itemize}
    \item $x$
    \item $y$
    \item $z$
    \item pitch (angle 1): $\theta$
    \item roll (angle 2): $\phi$
    \item yaw (angle 3): $\gamma$
    \item focal length: $f$
\end{itemize}

\section{Transform Idea}
Keep track of how the camera moves with user input, and use this to convert world points to view plane points.
\newline\newline Taking $f > 0$ and all other values 0 to correspond to the camera facing down the positive z-axis, with x and y axes normalized, placed at $(0,0,0)$ with view plane centered at $(0, 0, f)$ camera coordinates, parallel to the xy-plane. So the view
\newline\newline \textbf{How to convert from (x, y, z) in the world to (X, Y) on the screen?}

    \texttt{StdDraw.point(xClip, yClip)}
    \[worldPoint =  \begin{pmatrix}
    x \\ y \\ z
    \end{pmatrix}\]
    \[T(worldPoint) = \begin{pmatrix}
    \text{xClip} \\ \text{yClip}
    \end{pmatrix}\]
Find $T$ that converts from world coordinates to clipping coodinates.
\[T: WorldSpace \longrightarrow CameraSpace \longrightarrow ClipSpace\]
Figure out which way to apply the transformations from in matrices.


\end{document}